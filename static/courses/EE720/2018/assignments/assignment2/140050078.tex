% EE720 Assignment for 140050078

\documentclass[12pt,a4paper,answers]{exam}
\usepackage{mathtools}
\usepackage{amsmath,amsfonts,amssymb}
\usepackage{hyperref}

\addpoints%
\bracketedpoints%

\begin{document}
\pagestyle{head}
\headrule
\lhead[Assignment 2: 10 \ \points]{}
\chead[
EE 720: Introduction to Number Theory and Cryptography (Spring 2018)\\ 
Instructor: Saravanan Vijayakumaran\\
Indian Institute of Technology Bombay\\
]{}
\rhead[Date: January 23, 2018]{}%

\cfoot[]{\thepage\ of \numpages}
\extraheadheight{.4in}%
\extrafootheight{-0.5in}%
\extrawidth{0.5in}%

Find the pdf file corresponding to your roll number in the directory \url{https://www.ee.iitb.ac.in/~sarva/courses/EE720/2018/assignments/assignment2/}. Upload the answers as a \textbf{pdf} file in Moodle. Use the tex file provided in the directory to fill in your answers. The \textbf{upload deadline} will be 11:00pm IST on Wednesday, January 31, 2018.
\begin{questions}
\question[5] Prove that the Vigen\'ere cipher using period $t$ is perfectly indistinguishable when used to encrypt messages of length $t$. Prove this directly without proving the perfect secrecy of the scheme and then using the equivalence of perfect secrecy and perfect indistinguishability.
  \begin{solution}
    Write your answer here
  \end{solution}

  \question[5] State whether the following encryption scheme is perfectly secret or not. Justify your answer either with a proof or a counterexample.

  The message space is $\mathcal{M} = \{0,\ldots,4\}$. Algorithm \texttt{Gen} chooses a uniform key from the keyspace $\{0,\ldots,5\}$. $\texttt{Enc}_k(m) = (k+m) \bmod 5$ and $\texttt{Dec}_k(c) = (c-k) \bmod 5$.
  \begin{solution}
    Write your answer here
  \end{solution}

  
\end{questions}
\end{document}

