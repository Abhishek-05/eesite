% EE720 Assignment for 150020010

\documentclass[12pt,a4paper,answers]{exam}
\usepackage{mathtools}
\usepackage{amsmath,amsfonts,amssymb}
\usepackage{hyperref}

\addpoints%
\bracketedpoints%

\begin{document}
\pagestyle{head}
\headrule
\lhead[Assignment 2: 10 \ \points]{}
\chead[
EE 720: Introduction to Number Theory and Cryptography (Spring 2018)\\ 
Instructor: Saravanan Vijayakumaran\\
Indian Institute of Technology Bombay\\
]{}
\rhead[Date: January 23, 2018]{}%

\cfoot[]{\thepage\ of \numpages}
\extraheadheight{.4in}%
\extrafootheight{-0.5in}%
\extrawidth{0.5in}%

Find the pdf file corresponding to your roll number in the directory \url{https://www.ee.iitb.ac.in/~sarva/courses/EE720/2018/assignments/assignment2/}. Upload the answers as a \textbf{pdf} file in Moodle. Use the tex file provided in the directory to fill in your answers. The \textbf{upload deadline} will be 11:00pm IST on Wednesday, January 31, 2018.
\begin{questions}
\question[5] Prove that the Vigen\'ere cipher using period $t$ is perfectly indistinguishable when used to encrypt messages of length $t$. Prove this directly without proving the perfect secrecy of the scheme and then using the equivalence of perfect secrecy and perfect indistinguishability.
  \begin{solution}
    Write your answer here
  \end{solution}

  \question[5] When the one-time pad is used with the all-zeros key, i.e.~$k = 0^l$, we have $\texttt{Enc}_k(m) = m \oplus k = m$. This means that the plaintext will be sent as it is. To prevent this, suppose we modify the one-time pad to use only non-zero keys, $k \neq 0^l$. The key generation algorithm $\texttt{Gen}$ picks key $k$ uniformly from the set $\{0,1\}^l \setminus \{0^l\}$ which has cardinality $2^l - 1$. Is this modified scheme still perfectly secret? Justify your answer either with a proof or a counterexample.
  \begin{solution}
    Write your answer here
  \end{solution}

  
\end{questions}
\end{document}

